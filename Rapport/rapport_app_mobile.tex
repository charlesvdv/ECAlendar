\documentclass{article}

\usepackage[version=3]{mhchem} % Package for chemical equation typesetting
\usepackage{siunitx} % Provides the \SI{}{} and \si{} command for typesetting SI units
\usepackage{graphicx} % Required for the inclusion of images
\usepackage{natbib} % Required to change bibliography style to APA
\usepackage{amsmath} % Required for some math elements
\usepackage[utf8x]{inputenc}
\usepackage{hyperref}

\renewcommand{\labelenumi}{\alph{enumi}.} % Make numbering in the enumerate environment by letter rather than number (e.g. section 6)

%\usepackage{times} % Uncomment to use the Times New Roman font

%----------------------------------------------------------------------------------------
%	DOCUMENT INFORMATION
%----------------------------------------------------------------------------------------

\title{AM4L: Applications Mobiles \\ Réalisation d'une application Android (ECAlendar) \\ ECAM Bruxelles} % Title

\author{Charles \textsc{Vandevoorde} (13019)\\
		Lorenzo \textsc{Riga} (13018)\\
		Antoine \textsc{Vander Meiren} (12088)\\
		Gaetano \textsc{Giordano} (12054)\\
		Sylvain \textsc{Alonso} (12150)\\} % Author name

\date{\today} % Date for the report

\begin{document}

	\maketitle % Insert the title, author and date


	\section*{Introduction}
	\hspace{0.45cm}
	Dans le cadre du cours d'applications mobiles, nous avons réalisé une application Android "ECAlendar" à l'aide d'Android Studio. Cette dernière utilise l'API de 'Ecam Calendar" afin de permettre aux étudiants de consulter leurs horaires directement sur leurs GSM.\\

	Dans ce rapport, nous allons présenter la manière dont nous nous sommes organisés en groupe mais aussi la structure de notre application ainsi qu'un listing de ses fonctionnalités.

	\section{Organisation du groupe}
	\hspace{0.45cm}
	 Nous avons hébergé notre projet sur Github de manière à faciliter le travail en groupe et nous sommes répartis les tâches comme suit lors de la première séance:
	 \begin{itemize}
	 	\item Charles:
	 	\item Lorenzo:
	 	\item Antoine: Gestion des préférences, analyse est création des modèles requis et gestion de recherche de calendrier
	 	\item Gaetano:
	 	\item Sylvain: intégration de la base de données SQLite et gestion des clics (affichage des détails lorsque l'utilisateur appuie sur un élément).
	 \end{itemize}

	\section{Présentation de l'architecture (diagramme de classes)}
	\hspace{0.45cm}

	\section{Listing des fonctionnalités}
	\hspace{0.45cm}
		\begin{itemize}
			\item Affichage de calendrier reprennant son horraire personnel
			\item Recherche parmis les calendrier disponible via l'API de l'ECAM
			\item Affichage simultané d'un second calendrier au choix pour comparer des horaires (local, professeur et étudiant)
		\end{itemize}

	\section{Explication des différentes parties de l'applicaiton}
	\subsection{Création des modèles}
		Avant de commencer à développer l'application, nous avons du analyser ce qui est renvoyé par l'API proposé par ecam calendar
		afin de réaliser un modèle des données que nous allons devoir stocker. Nous avons décider de créer deux modèles différents :
		Schedule et calendarType. \
		Schedule stocke toutes les informations relative à un calendrier en particuler : Local, date et heure,
		professeur, note, nom de l'activité, personne concernée par groupe, ... \
		CalendarType quant à lui permet de sauver les différents calendrier disponible via un ID et une description du l'année/section,
		du professeur ou du local. \

		Tout au long de l'application nous utiliserons ces modèles pour créer des objets utilisables par chacune des parties.


	\subsection{Gestion des préférences}
		Pour ce qui est des préférences, nous avons simplement effectuer une préférence activity, accesible via le menu Android de l'application,
		et composée d'un fragment de préférence affichant une liste d'horraire disponible
		qui permet à l'étudiant de séléctionner son horaire personnel, en l'occurence l'année dans laquelle il se trouve et sa série. \
		Au premier lancement de l'application, l'activité princinpale n'affichera pas de calendrier et nous demandera de nous rendre dans
		l'onglet préférence afin de séléctionner l'année désirée. \
		A partir de ce menu, nous avons développer une "Up Navigation" qui permet de retrouner directement à la l'activité principale montrant
		le calendrier séléctionné. Notons que nous pouvons également utilisé le boutton back nativement présent sur les smartphone Android.

	\subsection{Recherche d'un second calendier}
		L'application permet à l'utilisateur de séléctionner deux calendrier à la fois : Le sien comme principale via les préférence et
		un second pour pouvoir comparer deux calendrier par exemple.  \
		Pour en séléctionner un second, nous avons ajouter une petite loupe dans le menu Android de l'application qui nous amène à une autre
		activité nommée searchActivity. Cette activité nous montre une liste, scrolable, de tous les calendrier disponible via l'API calender de l'ECAM.
		En clicant une seconde fois sur cette loupe nous faisons apparaitre un clavier qui permet de rentrer du texte afin de rechercher
		un calendrier particuler. \
		La recherche permettra à l'utilisateur de visualiser au fûr et à mesure qu'il tape, les calendriers disponible en fonction du
		texte qu'il complète afin de simplifier la séléction désirée. \\
		L'utilisateur pourra ensuite cliquer  sur le calendrier de son choix, il recevra une confirmation de ce choix via un message toast,
		et pourra retourner à l'activité principale via la "Up Navigation" ou encore le boutton back natif android. La valeur du calendrier
		sera alors stockée dans un sharedpréférence disponible pour le reste de l'applicaiton.
		Une fois sur l'acivité principale, le second calendrier sera disponible pour l'utilisateur. \
		A chaque redémarage de l'applicaiton ou changement de la section dans les préférence de l'application (onDestroy de la mainactivity),
		cette sharedPreference sera remise à zéro pour ne plus apparaitre dans le calendrier principale de l'utilisateur.

	\section{Conclusion}
	\hspace{0.45cm}
		

\end{document}
