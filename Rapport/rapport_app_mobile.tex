\documentclass{article}

\usepackage{graphicx} % Required for the inclusion of images
\usepackage[utf8x]{inputenc}
\usepackage{hyperref}

\title{AM4L: Applications Mobiles \\ Réalisation d'une application Android \\ ECAlendar \\
\vspace{1cm} \large{ECAM Bruxelles} \vspace{5cm}}

\author{Charles \textsc{Vandevoorde} (13019)\\
		Lorenzo \textsc{Riga} (13018)\\
		Antoine \textsc{Vander Meiren} (12088)\\
		Gaetano \textsc{Giordano} (12054)\\
		Sylvain \textsc{Alonso} (12150)\\}

\date{\vspace{5cm}\today}

\begin{document}

	\maketitle
    \newpage


	\section{Introduction}
    Dans le cadre du cours d'applications mobiles, nous avons réalisé une application Android
    "ECAlendar" à l'aide d'Android Studio. Cette dernière utilise l'API de "Ecam Calendar" afin de
    permettre aux étudiants de consulter leurs horaires directement sur leurs GSM.\\

    Dans ce rapport, nous allons présenter la manière dont nous nous sommes organisés en groupe mais
    aussi la structure de notre application ainsi qu'un listing de ses fonctionnalités.

	\section{Organisation du groupe}
	\hspace{0.45cm}
     Nous avons hébergé notre projet sur Github de manière à faciliter le travail en groupe et nous
     sommes répartis les tâches comme suit lors de la première séance:
	 \begin{description}
         \item[Charles] chargement des données depuis l'API, DAO ainsi que gestion de la
             \textit{RecyclerView}.
         \item[Lorenzo] gestion du layout, style de l'application.
         \item[Antoine]
         \item[Gaetano] gestion du layout et style de l'application
         \item[Sylvain] intégration de la base de données SQLite et gestion des clics (affichage des détails lorsque l'utilisateur appuie sur un élément).
	 \end{description}


	\section{Présentation de l'architecture}
        \begin{figure}
            \centering
            \includegraphics[scale=0.2]{img/uml.png}
            \label{fig:uml}
            \caption{Diagramme UML de classe représentant notre application}
        \end{figure}

        \subsection{Récupération des données depuis l'API REST}
            Notre application pour fonctionner a besoin de données provenant de l'API REST du site
            \url{calendar.ecam.be}. Après une analyse des requêtes faites par l'application web
            Calendar, nous avions tous les outils en main pour démarrer l'implémentation de l'API
            dans notre application.

            Pour éviter de bloquer l'interface utilisateur pendant le chargement des données, nous
            avons utilisé un \textit{IntentService}. Celui-ci est un service spécial d'Android qui
            s'arrête dès que la tâche est finie. Nous avons choisi ce type de service à la place de
            système comme les \textit{AsyncTaskLoader}, \ldots pour plusieurs raisons.
            \begin{itemize}
                \item Meilleur séparation des affaires puisqu'un \textit{IntentService} ne doit pas
                    obligatoirement être appelé de l'activité principale.
                \item API plus propre grâce à l'utilisatation d'\textit{Intent}.
            \end{itemize}

            Notre service peut être appelé pour récuperer deux types de données: la liste des
            calendriers disponibles (profs, locals, séries, \ldots) et un calendrier. Pour chaque
            type, nous envoyons la requête via un \textit{Intent} contenant les données nécessaires
            à la récupération des données.

            Les requêtes HTTP sont faites synchrone à l'aide de \textit{future} puisqu'Android ne
            supporte pas le multi-threading imbriqué. Pour ce faire, nous avons utilisé la librairie
            \textit{Volley} qui est une nouvelle libraire permettant de faciliter les requêtes HTTP.

            L'utilisation des \textit{futures} vient du fait que \textit{Volley} est asynchrone de
            base.

            Le parsing des données ICS (format de donnée de calendrier) est effectué par une
            libraire appelée \textit{iCal4j}. Le parsing est effectué dans l'\textit{IntentService}
            pour éviter des ralentissements inutiles de l'interface utilisateur.

            Finalement, les données sont renvoyés via un \textit{Broadcast} jusqu'à un récepteur
            spécialement écrit pour ce service. Nous détaillerons plus en détail ce récepteur dans
            la section \ref{sec:dao}.

        \label{sec:dao}
        \subsection{Le DAO}
            DAO signifie \textit{Data Access Object}. Cette classe permet d'être le seul point
            d'accès aux données et décide de quel source les données seront tirées. Dans notre cas,
            les sources sont la base de donnée et l'API REST du site \url{calendar.ecam.be}.

            La sélection de la source est assez trivial, nous regardons d'abord si les données sont
            en base de données et si elles sont assez nouvelles (moins de 1 semaines) aussi non,
            nous allons utilisés notre \textit{IntentService} pour charger nos données et les sauver
            dans la base de donnée.

            Mise à part la création, le DAO est responsable de la sauvegarde et de la reprise de
            données en base de données.

            Le DAO devant gérer l'asynchronisme de l'\textit{IntentService}, nous avons utilisés un
            système de "notifieurs" qui demande d'être notifié dès qu'une donnée est arrivée. Le DAO
            garde donc une liste de classe implémentant une certaine interface.

            Une autre caractéristique du DAO est que celui-ci est implémenté comme un singleton
            parce que nous devions avoir le contexte d'une activité pour sauvegardé les données en
            base de données. Lorsque le \textit{CalendarLoaderReceiver} (le récepteur de
            l'\textit{IntentService}) reçoit les données, nous devons accèder au DAO mais le
            récepteur n'a pas de contexte. Il faut donc avoir un context statique (mauvaise pratique
            en général) ou alors implémenter le DAO entant que singleton. Cette décision nous a
            permis également d'être beaucoup plus fexible quand au chargement des données.

        \subsection{Le \textit{RecyclerView}}
            Le \textit{RecyclerView} nous permet d'afficher une grande liste de calendriers sans
            problème de performances. Dans notre cas, nous avons créer une \textit{RecyclerView}
            avec deux types de données. Nous pouvions soit avoir un slot horaire, soit avoir une
            information sur la date des slots horaires suivant.

            Dans notre \textit{ItemAdapter} qui gère la création à la volée des informations
            affichées, nous avons donc deux différents \textit{ViewHolder}.

            La \textit{RecyclerView} ne supportant qu'une seule liste de donnée, nous avons créer
            une liste d'\textit{Object}. Cette approche pourrait être changé dans le futur si nous
            rajoutons plusieurs types de vues différentes. Dans notre cas, cette solution est la
            plus facile et la plus rapide à implémenter.

	\section{Listing des fonctionnalités}
	
	\subsection{Icon}
	 Que serait une application mobile, sans une icon appropri\'ee. C'est pour cela que nous avons cr\'ee une icon ECAlendar specialement customis\'ee pour notre application.
	   \begin{center}
            \includegraphics[scale=0.4]{img/ICON.png}
            \end{center}
            
            
            
	 \subsection{La page d'accueil}
	 Comme vous pouvez le voir sur l'image, en entrant dans l'application ECAlendar nous avons une page qui nous indique les cours pour l'ann\'ee s\'electionn\'ee. Autrement dit elle indique la date ainsi que les cours donn\'es. Elle indique \'egalement l'heure du debut et l'heure de fin du cours.
	   \begin{center}
            \includegraphics[scale=0.4]{img/Page_accueil.png}
            \end{center}
            

	 \subsection{D\'etail d'un cours}
	 En cliquant sur l'un des cours on entre dans une nouvelle page qui nous permet d'afficher les d\'etails concernant ce cours. C'est a dire le local, l'enseignant et la s\'erie.
	 
            \begin{center}
            \includegraphics[scale=0.4]{img/Section.png}
            \end{center}
            
            \subsection{Choix de l'ann\'ee}
	Bien evidement, notre application nous permet de choisir \'egalement l'ann\'ee d'\'etude pour lequel nous aimerions voir l'horaire. C 'est pour cela que sur la page d'accueil en  cliquant sur les 3 points  en haute \`a droite, cela nous redirigera vers une page qui nous permettra de cocher une ann\'ee et ainsi la mettre dans nos pr\'eferences.
	 
            \begin{center}
            \includegraphics[scale=0.4]{img/Annee.png}
            \end{center}

	\subsection{Recherche}
	Nous pouvons \'egalement choisir l'ann\'ee sur la page d'accueil, en cliquant simplement sur la petite loupe situ\'ee en haut. Cela nous redirigera vers la page ci-dessous.
            \begin{center}
            \includegraphics[scale=0.4]{img/search.png}
            \end{center}
            
	
	

	\section{Conclusion}

\end{document}
